% Chapter 1

\chapter{Concluding Discussion} % Main chapter title
\captionsetup{justification=raggedright,singlelinecheck=false}

\label{Conclusion}  

\lhead{Concluding Discussion. \emph{Conclusion}} % This is for the header on each page - perhaps a shortened title

A parity-violating scattering asymmetry of longitudinally polarized electrons from unpolarized protons was determined using approximately 2/3 of the \Qs dataset. The asymmetry, measured to be  $A_{PV}=-221.5\pm9.3\text{(stat)}\pm3.1\text{(sys)~ppb}$, remains blinded due to ongoing analysis. This represents a 5\% measurement of the parity-violating asymmetry. Assuming that the final result includes a dataset 1.5 times larger, one could expect the full result to have a statistical error on the asymmetry of 7.6~ppb. A similar increase in systematic error in the added dataset gives a total error of $\delta A/A\approx$4.1\%. Although this falls short of the original proposal, it represents the most precise measurement of the parity-violating electron scattering asymmetry ($A_{PV}$) ever made. When unblinded, this result will test the Standard Model prediction of the weak charge of the proton. This measured parity-violating asymmetry using 2/3 of the full dataset translates into a $\pm$7.8\% measure of the proton's weak charge. The blinding term on the asymmetry is somewhere in the $\pm$60~ppb range, translating into a range for the proton weak charge from 0.045 to 0.096. This range allows for as large as a 4.7$\sigma$ deviation from the Standard Model.

Insights gained during the \Qs experiment will impact the future parity program going forward. Reduction of key systematic errors in Compton polarimetry achieved during \Q, make the stringent polarimetry requirements of future experiments such as SOLID and MOLLER appear to be achievable. Efforts underway to understand the processes giving rise to the observed ``PMT double-difference'' will influence the design of future parity experiments. 

Experimental physics at the precision frontier is expected to be challenging and, in this regard, it never seems to disappoint; however, the potential for discovery makes it worth the effort.
